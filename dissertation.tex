%% LaTeX template for BSc Computing for Games final year project dissertations
%% by Edward Powley
%% Games Academy, Falmouth University, UK

%% Based on:
%% bare_jrnl.tex
%% V1.4b
%% 2015/08/26
%% by Michael Shell
%% see http://www.michaelshell.org/
%% for current contact information.
%%
%% This is a skeleton file demonstrating the use of IEEEtran.cls
%% (requires IEEEtran.cls version 1.8b or later) with an IEEE
%% journal paper.
%%
%% Support sites:
%% http://www.michaelshell.org/tex/ieeetran/
%% http://www.ctan.org/pkg/ieeetran
%% and
%% http://www.ieee.org/

%%*************************************************************************
%% Legal Notice:
%% This code is offered as-is without any warranty either expressed or
%% implied; without even the implied warranty of MERCHANTABILITY or
%% FITNESS FOR A PARTICULAR PURPOSE! 
%% User assumes all risk.
%% In no event shall the IEEE or any contributor to this code be liable for
%% any damages or losses, including, but not limited to, incidental,
%% consequential, or any other damages, resulting from the use or misuse
%% of any information contained here.
%%
%% All comments are the opinions of their respective authors and are not
%% necessarily endorsed by the IEEE.
%%
%% This work is distributed under the LaTeX Project Public License (LPPL)
%% ( http://www.latex-project.org/ ) version 1.3, and may be freely used,
%% distributed and modified. A copy of the LPPL, version 1.3, is included
%% in the base LaTeX documentation of all distributions of LaTeX released
%% 2003/12/01 or later.
%% Retain all contribution notices and credits.
%% ** Modified files should be clearly indicated as such, including  **
%% ** renaming them and changing author support contact information. **
%%*************************************************************************


\documentclass[journal]{IEEEtran}

\usepackage{graphicx}
% Insert additional usepackage commands here

\begin{document}
%
% paper title
% Titles are generally capitalized except for words such as a, an, and, as,
% at, but, by, for, in, nor, of, on, or, the, to and up, which are usually
% not capitalized unless they are the first or last word of the title.
% Linebreaks \\ can be used within to get better formatting as desired.
% Do not put math or special symbols in the title.
\title{How does visualising RRT pathfinding in an AI agent effect the perceived intelligence of the agent?}
%
%
% author name
\author{1507866}

% The paper headers -- please do not change these, but uncomment one of them as appropriate
% Uncomment this one for COMP320
\markboth{COMP320: Research Review and Proposal}{COMP320: Research Review and Proposal}
% Uncomment this one for COMP360
% \markboth{COMP360: Dissertation}{COMP360: Dissertation}

% make the title area
\maketitle

% As a general rule, do not put math, special symbols or citations
% in the abstract or keywords.
\begin{abstract}
The abstract goes here.
\end{abstract}

\section{Introduction}
% The very first letter is a 2 line initial drop letter followed
% by the rest of the first word in caps.
% 
% form to use if the first word consists of a single letter:
% \IEEEPARstart{A}{demo} file is ....
% 
% form to use if you need the single drop letter followed by
% normal text (unknown if ever used by the IEEE):
% \IEEEPARstart{A}{}demo file is ....
% 
% Some journals put the first two words in caps:
% \IEEEPARstart{T}{his demo} file is ....
% 
% Here we have the typical use of a "T" for an initial drop letter
% and "HIS" in caps to complete the first word.
\IEEEPARstart{T}{his} demo file is intended to serve as a ``starter file''
for your final year project dissertation.
\section{Research Question}
\subsection{What is (are) the key research question(s) that you will seek to answer in	your project?}
How does visualising RRT path finding in an AI agent effect the perceived intelligence of the agent?

\subsection{How will answering these questions contribute to the state of knowledge in the field of your project?}
Papers have looked at visualising and foregrounding AI but not at what effect this has on the player's perception of how intelligent the agent is.

The question proposed in this paper is: how does visualising RRT path finding in an AI agent effect the perceived intelligence of the agent in digital games?  

While visualising and foregrounding AI in games has been looked at previously \cite{} the effects on perceived intelligence have not been researched??

The aim of the paper is visualise RRT path-finding on an AI agent in a game .....


\subsection{Hypothesis:}
Null: Visualising RRT has no effect on perceived intelligence of the AI.


\section{Literature Review}
\subsection{Visualising Data/AI}

Haworth \textit{et al} research visualising decision trees in games to see what effect it had on children's analytical reasoning and game play \cite{Haworth2010}. 

While they did not come any definite conclusions their results suggested that data aided players in playing the game. However an issue they noted was that the game could be unbalanced at the end making the usefulness of the tree being displayed questionable.  

Haworth \textit{et al} only made a simple game that was tested on children. In contrast Isla visualised pathfinding in a game that is now for sale?? (Word it better) \cite{Isla2014}.
 
 
 CHECK Cook \textit{et al} surveyed many games that foreground AI and proposed design patterns for different methods of foregrounding AI.  The method proposed in this paper is similar to their design pattern Visualising AI.  
 Third Eye Crime is a game that followed this design pattern \cite{Isla2014}.  Third Eye Crime displayed the enemy's path finding to the player using Occupancy maps. 
 
 This was designed to make the player want trigger the mechanic ...  Similarly the pathfinding visualisation  


\subsection{Pathfinding}
Third Eye Crime \cite{Isla2014} visualises enemy path finding as the main mechanic. Isla uses occupancy maps this does not produced an exact path but shows the probability of the players being in an area. 

FIND NAME \textit{et al} surveyed numerous papers on path finding. 

FIND NAME looked at path finding in 3D while the paper applied to planes it may be relevant here...

PAPER ON RRT



\section{Methodology}
\subsection{What methodology will you use to seek answers to these questions?}
The methodology will be involve human participants who will play one variation of the game. The four variations are firstly a version of the game with no visible pathfinding where the enemy will follow a pre defined route. The second variation will have a visual tree in front of the enemy but it will be random and not seeking the participants. Thirdly will use RRT path finding to seek out and move towards the participant. Finally will be a version that always know the participant's position and is therefore always moving towards them.

While the participant is playing the game will export their location to a CSV file every second for use in R. There will also be a questionnaire for the participants to fill out after completing the play test.


\subsection{Preliminary Results}
What preliminary results have you obtained?


\section{Conclusion}
The conclusion goes here.

% references section

\bibliographystyle{IEEEtran}
\bibliography{references}

% Appendices

% \appendices
% \section{First appendix}
% Appendices are optional. Delete or comment out this part if you do not need them.

% that's all folks
\end{document}
