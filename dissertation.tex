%% LaTeX template for BSc Computing for Games final year project dissertations
%% by Edward Powley
%% Games Academy, Falmouth University, UK

%% Based on:
%% bare_jrnl.tex
%% V1.4b
%% 2015/08/26
%% by Michael Shell
%% see http://www.michaelshell.org/
%% for current contact information.
%%
%% This is a skeleton file demonstrating the use of IEEEtran.cls
%% (requires IEEEtran.cls version 1.8b or later) with an IEEE
%% journal paper.
%%
%% Support sites:
%% http://www.michaelshell.org/tex/ieeetran/
%% http://www.ctan.org/pkg/ieeetran
%% and
%% http://www.ieee.org/

%%*************************************************************************
%% Legal Notice:
%% This code is offered as-is without any warranty either expressed or
%% implied; without even the implied warranty of MERCHANTABILITY or
%% FITNESS FOR A PARTICULAR PURPOSE! 
%% User assumes all risk.
%% In no event shall the IEEE or any contributor to this code be liable for
%% any damages or losses, including, but not limited to, incidental,
%% consequential, or any other damages, resulting from the use or misuse
%% of any information contained here.
%%
%% All comments are the opinions of their respective authors and are not
%% necessarily endorsed by the IEEE.
%%
%% This work is distributed under the LaTeX Project Public License (LPPL)
%% ( http://www.latex-project.org/ ) version 1.3, and may be freely used,
%% distributed and modified. A copy of the LPPL, version 1.3, is included
%% in the base LaTeX documentation of all distributions of LaTeX released
%% 2003/12/01 or later.
%% Retain all contribution notices and credits.
%% ** Modified files should be clearly indicated as such, including  **
%% ** renaming them and changing author support contact information. **
%%*************************************************************************


\documentclass[journal]{IEEEtran}

\usepackage{graphicx}
% Insert additional usepackage commands here

\begin{document}
%
% paper title
% Titles are generally capitalized except for words such as a, an, and, as,
% at, but, by, for, in, nor, of, on, or, the, to and up, which are usually
% not capitalized unless they are the first or last word of the title.
% Linebreaks \\ can be used within to get better formatting as desired.
% Do not put math or special symbols in the title.
\title{ How Does Visualising RRT Pathfinding in an NPC Effect the Perceived Intelligence of the NPC?}
%
%
% author name
\author{1507866}

% The paper headers -- please do not change these, but uncomment one of them as appropriate
% Uncomment this one for COMP320
\markboth{COMP320: Research Review and Proposal}{COMP320: Research Review and Proposal}
% Uncomment this one for COMP360
% \markboth{COMP360: Dissertation}{COMP360: Dissertation}

% make the title area
\maketitle

% As a general rule, do not put math, special symbols or citations
% in the abstract or keywords.
\begin{abstract}
The abstract goes here.
\end{abstract}

\section{Introduction}
% The very first letter is a 2 line initial drop letter followed
% by the rest of the first word in caps.
% 
% form to use if the first word consists of a single letter:
% \IEEEPARstart{A}{demo} file is ....
% 
% form to use if you need the single drop letter followed by
% normal text (unknown if ever used by the IEEE):
% \IEEEPARstart{A}{}demo file is ....
% 
% Some journals put the first two words in caps:
% \IEEEPARstart{T}{his demo} file is ....
% 
% Here we have the typical use of a "T" for an initial drop letter
% and "HIS" in caps to complete the first word.
\IEEEPARstart{I}{troduction} section.... \\
The research questions proposed in this project are: how does visualising RRT path finding in a Non Player Character (NPC) effect the perceived intelligence of the NPC in digital games? and how does visualising RRT path finding in a Non Player Character (NPC) effect the way partcipants navigate a level of game? 

The project will look at different methods of visualising RRT path finding to investigate what effects that can have on how the partcipants plays the game and explores the level.  Previous papers have researched visualising Artifical Intelligence (AI) and foregrounding AI but not at what effect this has on how the partcipants play the game.


\subsection{Hypothesis:}
\textbf{Null}: Visualising RRT has no effect on how the partcipant plays a level of game. \\
\textbf{Hypothesis}: Visualising RRT has a significant effect on how the partcipant plays a level of game. 


\section{Literature Review}
\subsection{Foregrounding and Visualising AI}
While AI is frequently used in digital games it is not often visualised or foregrounded. However, Treanor \textit{et al} say that the AI is often designed to fit the game and is therefore rudimentary??? \cite{treanor2015}.
They surveyed many games that foreground AI in different ways and proposed design patterns for different methods of foregrounding AI. The two design patterns of interest for this project are AI as a Villain and AI is Visualised. The design pattern "AI as Villian" is where the enemy's AI is trying to create an experience instead or out right defeating the player. The example they gave as an example of this pattern was Alien Isolation where the player is hunted by an NPC.  This related the this paper as the visualising of RRT is not to find a way to find the player but to give the player a way to predict or interpret what the Ai is doing and therefore overcome it.
Another design pattern of interest is "AI is Visualised". This pattern involves visualising the AI's state and decision making. This is normally hidden from the player but this design pattern looks at making a mechanic from visualising it. The example given by Treanor \textit{et al} is Third Eye Crime.  Third Eye Crime is a game that followed the "AI is Visualised" design pattern \cite{Isla2014}.  Third Eye Crime displayed the enemy's path finding to the player using Occupancy maps.  
This was designed to make the player want trigger the mechanic ...  Again this paper will also use this design pattern by visualising the NPC's path-finding.  The RRT path-finding will be visualised in different ways.


While Haworth \textit{et al} do not visualise an AI process they do visualise the possible decision in a game on a tree structure \cite{Haworth2010}. They research visualising decision trees in a game to see what effect it had on children's analytical reasoning and game play.  While they did not come to any definite conclusions their results suggested that data aided players in playing the game as in later level the children struggled to beat the game without the visualised tree. However, an issue they noted was that the game could be unbalanced at the end making the usefulness of the tree being questionable.  

A further issue is that Haworth \textit{et al} only tested the tree in a relatively simple 2D game that was tested on children. This does not give any data on 3D games on the market??? In contrast, Isla's visualised path-finding in Third Eye Crime is on sale?? (Word it better) \cite{Isla2014}.
 
Like  Haworth \textit{et al}, Bauer \textit{et al} also research visualising tree structures \cite{bauer2012}. However, they did use an AI technique, they used Rapidly-Exploring Random Trees (RRT). They used RRT in level design tools to predict possible moves the player could make.  They then used a clustering algorithm to organise the tree to make it legible? 

* how to present data to player 


\subsection{Pathfinding}
In digital games the A* path finding algorithm appears to be the most widely used \cite{Algfoor2015}.  Algfoor \textit{et al} surveyed numerous papers on path finding. The focus appeared to be on the type of grids used in path-finding and then numerous algorithms that can be used \cite{Algfoor2015}. The most popular being the A* algorithm for use in digital games and robotics. RRT path-finding was not mentioned. 
They surveyed many grid types and gave the advantages of each. However, RRT does not use grids it instead uses nodes making the grid type irrelevant.\\

Third Eye Crime was previously mentioned for it's use of design patterns. However it also uses path-finding as an important mechanic \cite{Isla2014}. Isla uses occupancy map for path-finding. Occupancy or Influence maps do not produce an exact path instead they show the probability of the player being in different parts of the map \cite{Isla2014, Miles2006}. Isla used Occupancy maps to show where the enemy AI thinks the player currently is. Miles and Loius used influence maps but used it to inform A* path-finding instead of use the map itself for path-finding \cite{Miles2006}.\\
 
A further paper on path finding is Wang and Lu's paper which looks at path finding in a 3D environment. While again they were using A* they look at using A* in 3D and suggest using nodes instead of a grid?? \cite{wang2012}.

Rapidly-Exploring Random Trees(RRT) are a search method used more widely in robotics than digital games\cite{Kuffner2000}. Kuffner and LaValle first proposed RRT in 2000, they intended to produce a random algorithm more efficient than the other search algorithms available at the time. RRT Path Planner is a variant of RRT proposed by Kuffner and LaValle that can be used to find paths from the generated tree \cite{Kuffner2000}.



\section{Methodology}
The methodology that will be used to seek the answers to the proposed questions will be play testing and questionnaires. This will require human participants to play the game and fill in the questionnaires. 

\subsection{Playtest Variations}
There will be multiple variations of the game. The first will have no visualisation the NPC will use path-finding to patrol the level but the will be no visualisation to indicate what it is doing.  The second variation will have a visualisation of the RRT path-finding in front of it. While the RRT will be used to path-find in a large area on a small area will be visualised in an attempt to not confuse the participants. The third variation will also have visualised RRT path-finding but instead of a visual tree it will be environmental queues that give the participants clues to where the NPC may go. A-B testing will used on participants. Participants will be assigned different version of the game to play and then the results will be compared. \\
\\
While the participant is playing the game will export their current location to a CSV file every second for use in R. A heat map can then be generated from this data to see if there are any significant patterns. A heat map will be used as they are easy to generate and easy to discern patterns from \cite{Wallner2015}

There will also be a questionnaire for the participants to fill out after completing the play test.

\subsection{Questionnaire}
The questionnaire will be completed using an online questionnaire such as Google Forms. The participants will have to answers questions using Likert scales.




\subsection{Preliminary Results}
What preliminary results have you obtained? \\
No preliminary experiments have been therefore there are no preliminary results yet.


\section{Conclusion}
The conclusion goes here.

% references section

\bibliographystyle{IEEEtran}
\bibliography{references}

% Appendices

% \appendices
% \section{First appendix}
% Appendices are optional. Delete or comment out this part if you do not need them.

% that's all folks
\end{document}
